\documentclass[a4paper]{article}
\usepackage{german}
\usepackage{fontspec}
\defaultfontfeatures{Mapping=tex-text,Scale=MatchLowercase}
\setmainfont{Raleway}
\usepackage{hyperref}
\usepackage{pdfpages}
\usepackage{fancyhdr}
\usepackage{tabularx}

\pagestyle{fancy}
\lhead{\includegraphics[scale=0.2,trim={5.5cm 16.5cm 5.5cm 16.5cm},clip]{logo.pdf}}
\rhead{HaSi e.V.\\ Effertsufer 104\\ 57072 Siegen}
\cfoot{}
\setlength{\headheight}{42pt} 

\renewcommand\LayoutTextField[2]{#1  \raisebox{-.30\baselineskip}{#2}}
\renewcommand\LayoutCheckField[2]{#1 \raisebox{-.30\baselineskip}{#2}}

\renewcommand{\headrulewidth}{0pt} %keine Trennlinie

\newcommand{\lowerElement}[1] {
  \raisebox{-.30\baselineskip}{#1}  
}



\begin{document}
\hspace{1cm}

\section*{Reisekostenabrechnung HaSi e.V.}

\subsection*{Persönliche Daten}
\begin{Form}[]
\begin{tabularx}{\textwidth}{p{3.2cm}  p{8.3cm}}
Vor- und Nachname: & \TextField[name=name,width=8.2cm, bordercolor=0 0 0]{}\\
IBAN: & \TextField[name=IBAN, width=8.2cm, bordercolor=0 0 0]{}\\
Name der Bank: &  \TextField[name=bankname, width=8.2cm, bordercolor=0 0 0]{}\\
Hasi-Nummer: &  \TextField[name=nummer, width=8.2cm, bordercolor=0 0 0]{} \\
\end{tabularx}
\end{Form}

\subsection*{Reise-Details}
\begin{Form}[]
\begin{tabularx}{\textwidth}{p{3.2cm} p{3.2cm} p{.95cm} p{3.2cm} }
Beginn: & 
\TextField[name=ReiseBeginn, bordercolor=0 0 0, value=Datum, width=3.2cm]{} &
Ende: &
\TextField[name=ReiseEnde, bordercolor=0 0 0, value=Datum, width=3.2cm]{} \\
\end{tabularx}
\begin{tabularx}{\textwidth}{p{3.2cm}  p{8.3cm}}
Start: & \TextField[name=ReiseStart, bordercolor=0 0 0, value={Name der Stadt}, width=8.2cm]{} \\
Ziel: & \TextField[name=ReiseZiel, bordercolor=0 0 0, value={Name der Stadt},width=8.2cm]{} \\
Anlass: & \TextField[name=Anlass, bordercolor=0 0 0, value={z.B. Name Veranstaltung},width=8.2cm]{} \\
Entfernung: & \TextField[name=Anlass, bordercolor=0 0 0, value={Entfernung in Km},width=8.2cm]{}\\
\end{tabularx}
\begin{tabularx}{\textwidth}{p{3.2cm} p{2cm} p{3.2cm}  }
Verkehrsmittel: & 
Auto \CheckBox[name=member, bordercolor=black]{} &
ÖPNV \CheckBox[name=member, bordercolor=black]{} \\
\end{tabularx}
\end{Form}

\subsection*{Erstattung}
\begin{Form}[]
\begin{tabularx}{\textwidth}{p{3.2cm}  p{8.3cm}}
Betrag: & \TextField[name=Betrag, bordercolor=0 0 0, width=8.2cm]{}
\end{tabularx}
\end{Form}

\noindent Der zu erstattende Betrag berechnet sich wie folgt:
\begin{itemize}
\item Bei ÖPNV: Summe der Kosten der Fahrkarten, \bf{maximal aber 50 €}.
\item Bei Auto: 0,25 € pro gefahrenem Km, \bf{maximal aber 50 €}.
\end{itemize}


\noindent Falls du mit ÖPNV gefahren bist, lege bitte deine Fahrkarten bei. Wenn du mit dem
eigenen Auto gefahren bist, ist eine Tank-Quittung \bf{nicht} notwendig.
\vspace{2.5cm}

\noindent\rule{\textwidth}{1pt}
Datum \hspace{1.5cm} Unterschrift Resende/r
\vspace{2.5cm}

\noindent\rule{\textwidth}{1pt}
Datum \hspace{1.5cm} Unterschrift Vorstand








\end{document}
